\chapter{Coarse-graining}
\label{chapter:method} 


\section{Introduction}

\begin{itemize}

\item{The idea of coarse graining in brief}

\end{itemize}

\section{Relationship with Fourier spectrum}

\begin{itemize}

\item{Show random noise with spectrum that top hat can resolve, compare spectra}

\item{Introduce integral relation between coarse-grain and Fourier spectrum}

\item{Show random noise top hat cannot resolve to motivate order}

\end{itemize}

\section{Concept of order}

\begin{itemize}

\item{Define order of a kernel}

\item{show higher order kernel difference in resolving one from previous section}

\item{Show how order can be used to show that will tend to the right slope}

\end{itemize}

\section{Extension to the sphere}

\begin{itemize}

\item{Just explain how concepts generalise to the sphere}

\end{itemize}

\section{Extension to vector fields}

\begin{itemize}

\item{Simple explanation of why Helmholtz is so useful}

\end{itemize}

\section{The coarse-grain energy flux}

\section{Note on the relationship with wavelets}


