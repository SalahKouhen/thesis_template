\chapter{Spectra in the atmosphere}
\label{chapter:literature_review} 


\section{Introduction}

\section{Spectra in turbulence theory}

\subsection{Kolmogorov's -5/3 spectrum}

\begin{itemize}

\item{Simple explanation for the existence of the scaling range}

\item{The log correction}

\item{Observation of the law}

\item{Simulation}

\end{itemize}

\subsection{Spectra in 2D turbulence}

\begin{itemize}

\item{Simple explanation for scaling ranges and directions of energy flux}

\item{Log corrections?}

\item{Observations of the law}

\item{Simulations of the law}

\end{itemize}

\subsection{Spectra and predictability}

\begin{itemize}

\item{Simple explanation of why energy at a scale is associated with the predictability time}

\item{Explanation of the finite time barrier of predictability}

\end{itemize}

\subsection{Methods of calculating the spectrum}

\begin{itemize}

\item{2D Fourier transform, exact formula and fact that you use Parseval and so square after}

\item{1D lines and the translation between them and the spectra, highlight importance}

\item{Structure functions and their link to the spectrum}

\item{Spherical harmonics and the translation between degree and wavenumber}

\item{Mention of wavelet techniques, this is a subcategory of coarse-graining which is addressed in detail in the second chapter and will not be expanded on here}

\end{itemize}

\section{Spectra in observations}

\subsection{First observations and historical context}

\begin{itemize}

\item{Nastrom and Gage}

\item{The idea of the energy gap and implications for predictability}

\end{itemize}

\subsection{Current observations}

\begin{itemize}

\item{The variablity of the slope with height, include that figure you made from the analysis that was novel in its presentation and cite all that one paper that had change with height}

\item{The direction of energy flux and why it matters}

\item{Variability rather than universality}

\end{itemize}

\section{Spectra in simulations}

\subsection{The effects of resolution}

\subsection{Aquaplanets and such}

\section{Theoretical explanations of the spectrum}

\subsection{The synoptic scale spectrum}

\begin{itemize}

\item{Charney and -3 and such}

\item{Other caveats}

\end{itemize}

\subsection{The mesoscale spectrum}

\subsubsection{Not 3D turbulence}

\subsubsection{2D inverse cascade}

\subsubsection{2D direct cascade}

\subsubsection{Wave wave interactions}

\subsubsection{Wave mean interactions}

\subsubsection{Direct imprinting}

\subsubsection{Stratified turbulence}

