\chapter{Proposed method 1}
\label{chapter:method_1} 


\section{Introduction}

\lipsum[2-4]

\section{Overview of the estimation process}

\lipsum[2-4]

\section{Methods}

\lipsum[2-4]

Example on dsplaying C/C++ source code. The argument "n" is often used to express the length of another input argument that follows it, such as:

\begin{lstlisting}[style=custom-c,caption={Function to balance a matrix.}]
extern PE_SIGPROC_API void pesig_balance(
        const size_t    n,
        signal_value_t  A[n][n],
        signal_value_t* S
        );
\end{lstlisting}

Example on displaying BASH/Console scripts:

\begin{lstlisting}[style=custom-bash,caption={A script in bash.}]
$ mkdir -p $HOME/code/pebase/realtime
$ cd $HOME/code/pebase/realtime
$ git clone git@github.com:maurovm/thesis_template.git repository
\end{lstlisting}

Example of displaying text as ``verbatim'' mode:

\begin{lstlisting}[style=custom-verbatim,caption={License information.}]
OxEngThesis is provided under:

    SPDX-License-Identifier:    GPL-2.0-only
    
OxEngThesis is free software: you can redistribute it or modify
it under the terms of the GNU General Public License as published by the
Free Software Foundation, version 2 only, according with:

    LICENSES/GPL-2.0

OxEngThesis is distributed in the hope that it will be useful, but
WITHOUT ANY WARRANTY; without even the implied warranty of MERCHANTABILITY
or FITNESS FOR A PARTICULAR PURPOSE. See the GNU General Public License
for more details.
\end{lstlisting}


    
\section{Results}

\lipsum[2-4]

\section{Discussion}

\lipsum[2-4]

\section{Conclusion}

\lipsum[2-4]